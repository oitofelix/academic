%
% Copyright (C) 2017 Bruno Félix Rezende Ribeiro <oitofelix@ufu.br>
%
% Copying and distribution of this file, with or without modification,
% are permitted in any medium without royalty provided the copyright
% notice and this notice are preserved.  This file is offered as-is,
% without any warranty.
%

\documentclass{article}

\usepackage[utf8]{inputenc}
% \usepackage[brazil]{babel}
\usepackage{amsthm}
\usepackage{amssymb}
\usepackage{mathtools}
\usepackage{cancel}
\usepackage{authblk}
\usepackage{hyperref}

\newtheorem{theorem}{Theorem}
\newtheorem{definition}[theorem]{Definition}
\newenvironment{proofNl}[1][\proofname]{\proof[#1]\mbox{}\\*}{\endproof}

\providecommand{\keywords}[1]{\textbf{\textit{Index terms---}} #1}
\newcommand{\Z}{\mathbb{Z}}
\newcommand{\Q}{\mathbb{Q}}
\newcommand{\R}{\mathbb{R}}
\newcommand{\norm}[1]{\left\|#1\right\|}
\newcommand{\ip}[2]{\left\langle #1, #2\right\rangle}
\newcommand{\function}[3]{#1:#2\rightarrow#3}
\newcommand{\comment}[2]{\stackrel{\mathclap{\textnormal{\tiny\mbox{#2}}}}{#1}}
% \newcommand{\boxedLeftArrow}{\raisebox{2pt}[\height][0pt]{
%     \framebox{\raisebox{-1pt}[\height][0pt]{\(\Leftarrow\)}}}\enspace}
% \newcommand{\boxedRightArrow}{\raisebox{2pt}[\height][0pt]{
%     \framebox{\raisebox{-1pt}[\height][0pt]{\(\Rightarrow\)}}}\enspace}


\title{Proof of Jordan-von Neumann theorem for vector spaces over \(\R\)}
\author{Bruno Félix Rezende Ribeiro \\ \href{mailto:oitofelix@ufu.br}{oitofelix@ufu.br}}
\affil{FAMAT --- Universidade Federal de Uberlândia}
\date{2017-04-15}

\begin{document}
\maketitle

\begin{abstract}
  The present work provides a proof of Jordan-von Neumann theorem for
  real vector spaces.  The proof presented here is a somewhat more
  detailed, particular case of the complex treatment given by
  P. Jordan and J. v. Neumann in \cite{JN}.
\end{abstract}

\keywords{Jordan-von Neumann Theorem, Vector spaces over \(\R\),
  Parallelogram law}

\begin{definition}[Inner Product]\label{def_ip}
  Let \(E\) be a vector space over \(\R\).  A function
  \(\function{\ip{.}{.}}{E \times E}{\R}\) is an \textbf{inner
    product} iff
  \begin{description}
  \item[P1]
    \(x \neq 0 \Rightarrow \ip{x}{x} > 0, \forall x \in E\)
  \item[P2]
    \(\ip{x}{y} = \ip{y}{x}, \forall x, y \in E\)
  \item[P3]
    \(\ip{x+y}{z} = \ip{x}{z} + \ip{y}{z}, \forall x, y, z \in E\)
  \item[P4]
    \(\ip{\lambda x}{y} = \lambda\ip{x}{y}, \forall\lambda\in\R,
    \forall x, y \in E\)
  \end{description}
\end{definition}

\begin{definition}
  A normed vector space \((E,\norm{.})\) satisfies the
  \textbf{parallelogram law} iff
  \[\norm{x+y}^2+\norm{x-y}^2=2(\norm{x}^2+\norm{y}^2), \forall x, y
    \in E\]
\end{definition}

\begin{theorem}
  Let \((E,\ip{.}{.})\) be an inner product vector space and define
  \mbox{\(\norm{x} = \sqrt{\ip{x}{x}}\)}.  Then, the normed vector
  space \((E,\norm{.})\) satisfies the parallelogram law.
\end{theorem}

\begin{proof}
  Given \(x, y \in E\), by definition \ref{def_ip},
  \begin{align*}
    & \norm{x+y}^2 + \norm{x-y}^2 \\
    & \comment{=}{\(\norm{.}\)} \ip{x+y}{x+y} + \ip{x-y}{x-y} \\
    & \comment{=}{P3} \ip{x}{x+y} + \ip{y}{x+y} + \ip{x}{x-y}
      + \ip{-y}{x-y} \\
    & \comment{=}{P2} \ip{x+y}{x} + \ip{x+y}{y} + \ip{x-y}{x}
      + \ip{x-y}{-y} \\
    & \comment{=}{P3} \ip{x}{x} + \ip{y}{x} + \ip{x}{y} + \ip{y}{y}
      + \ip{x}{x} + \ip{-y}{x} + \ip{x}{-y} + \ip{-y}{-y} \\
    & \comment{=}{P4} 2\ip{x}{x} + \cancel{\ip{y}{x}} + \ip{x}{y} + \ip{y}{y}
      - \cancel{\ip{y}{x}} + \ip{x}{-y} - \ip{y}{-y} \\
    & \comment{=}{P2} 2\ip{x}{x} + \ip{x}{y} + \ip{y}{y} + \ip{-y}{x} -
      \ip{-y}{y} \\
    & \comment{=}{P4} 2\ip{x}{x} + \ip{x}{y} + \ip{y}{y} - \ip{y}{x} +
      \ip{y}{y} \\
    & \comment{=}{P2} 2\ip{x}{x} + \cancel{\ip{x}{y}} + 2\ip{y}{y} -
      \cancel{\ip{x}{y}} \\
    & = 2(\ip{x}{x} + \ip{y}{y}) \\
    & \comment{=}{\(\norm{.}\)} 2(\norm{x}^2 + \norm{y}^2)
  \end{align*}
\end{proof}

\begin{theorem}
  Let \((E,\ip{.}{.})\) be an inner product vector space and define
  \mbox{\(\norm{x} = \sqrt{\ip{x}{x}}\)}.  Then,
  \[4\ip{x}{y} = \norm{x+y}^2 - \norm{x-y}^2\]
\end{theorem}

\begin{proof}
  Given \(x, y \in E\), by definition \ref{def_ip},
  \begin{align*}
    & \norm{x+y}^2 - \norm{x-y}^2 \\
    & \comment{=}{\(\norm{.}\)} \ip{x+y}{x+y} - \ip{x-y}{x-y} \\
    & \comment{=}{P3} \ip{x}{x+y} + \ip{y}{x+y} - \ip{x}{x-y}
      - \ip{-y}{x-y} \\
    & \comment{=}{P2} \ip{x+y}{x} + \ip{x+y}{y} - \ip{x-y}{x}
      - \ip{x-y}{-y} \\
    & \comment{=}{P3} \cancel{\ip{x}{x}} + \ip{y}{x} + \ip{x}{y} + \ip{y}{y}
      - \cancel{\ip{x}{x}} - \ip{-y}{x} - \ip{x}{-y} - \ip{-y}{-y} \\
    & \comment{=}{P4} \ip{y}{x} + \ip{x}{y} + \ip{y}{y}
      + \ip{y}{x} - \ip{x}{-y} + \ip{y}{-y} \\
    & \comment{=}{P2} 3\ip{x}{y} + \ip{y}{y} - \ip{-y}{x} + \ip{-y}{y} \\
    & \comment{=}{P4} 3\ip{x}{y} + \cancel{\ip{y}{y}} + \ip{y}{x} -
      \cancel{\ip{y}{y}} \\
    & \comment{=}{P2} 4\ip{x}{y}
  \end{align*}
\end{proof}

\begin{theorem}
  Let \((E,\norm{.})\) be a normed vector space which satisfies the
  parallelogram law, and define
  \[\ip{x}{y} = \frac{\norm{x+y}^2 - \norm{x-y}^2}{4}\]
  Then, \(\ip{.}{.}\) is an inner product.
\end{theorem}

\begin{proof}
  To prove this we verify each property of definition \ref{def_ip}.\\
  \\
  \fbox{P1} Given \(x \in E\) such that \(x \neq 0\),
  \begin{align*}
    \ip{x}{x}
    = \frac{\norm{x + x}^2 - \norm{x - x}^2}{4}
    = \frac{\norm{2x}^2 - \norm{0}^2}{4}
    = \frac{\cancel{4}\norm{x}^2}{\cancel{4}}
    = \norm{x}^2 > 0
  \end{align*}
  \\
  \fbox{P2} Given \(x, y, \in E\),
  \begin{align*}
    \ip{x}{y}
    &= \frac{\norm{x + y}^2 - \norm{x - y}^2}{4}
    = \frac{\norm{x + y}^2 - (|-1|\norm{x - y})^2}{4}\\
    &= \frac{\norm{x + y}^2 - \norm{(-1)(x - y)}^2}{4}
    = \frac{\norm{y + x}^2 - \norm{y - x}^2}{4}
    = \ip{y}{x}
  \end{align*}
  \\
  \fbox{P3} Given \(x, y, z \in E\), by the parallelogram law,
  \begin{align}
    & \norm{(x+z)+y}^2+\norm{(x+z)-y}^2=2(\norm{x+z}^2+\norm{y}^2)\label{eq1} \\
    & \norm{(x-z)+y}^2+\norm{(x-z)-y}^2=2(\norm{x-z}^2+\norm{y}^2)\label{eq2}
  \end{align}
  Subtracting equation \ref{eq2} from equation \ref{eq1},
  \begin{align}
    &\norm{(x+z)+y}^2 - \norm{(x-z)+y}^2 +\norm{(x+z)-y}^2 -
      \norm{(x-z)-y}^2 \nonumber\\
    &\qquad = 2(\norm{x+z}^2+\norm{y}^2) - 2(\norm{x-z}^2+\norm{y}^2) \nonumber\\
    \nonumber \\
    &\Rightarrow(\norm{(x+y)+z}^2 - \norm{(x+y)-z}^2) + (\norm{(x-y)+z}^2 -
    \norm{(x-y)-z}^2) \nonumber\\
    &\qquad = 2(\norm{x+z}^2 + \cancel{\norm{y}^2} - \norm{x-z}^2 -
    \cancel{\norm{y}^2}) \nonumber\\
    \nonumber\\
    &\Rightarrow 4\ip{x+y}{z} + 4\ip{x-y}{z} = 8\ip{x}{z} \nonumber\\
    \nonumber\\
    &\Rightarrow\ip{x+y}{z} + \ip{x-y}{z} = 2\ip{x}{z}\label{eq3}
  \end{align}
  Therefore, given \(x', y', z' \in E\) (by equation \ref{eq3}),
  \begin{align*}
    &\ip{\comment{\left(\frac{x'+y'}{2}\right)}{\normalsize\fbox{\(x\)}}
    +\comment{\left(\frac{x'-y'}{2}\right)}{\normalsize\fbox{\(y\)}}}{z'}
    +
    \ip{\comment{\left(\frac{x'+y'}{2}\right)}{\normalsize\fbox{\(x\)}}
    -\comment{\left(\frac{x'-y'}{2}\right)}{\normalsize\fbox{\(y\)}}}{z'}
    =
    2\ip{\comment{\left(\frac{x'+y'}{2}\right)}{\normalsize\fbox{\(x\)}}}{z'}
    \\\\
    &\comment{\Rightarrow}{P4}\ip{\frac{x'+\cancel{y'}+x'-\cancel{y'}}{2}}{z'} +
    \ip{\frac{\cancel{x'}+y'-\cancel{x'}+y'}{2}}{z'} = \ip{x'+y'}{z'}
    \\\\
    &\Rightarrow\ip{\frac{\cancel{2}x'}{\cancel{2}}}{z'} +
    \ip{\frac{\cancel{2}y'}{\cancel{2}}}{z'} = \ip{x'+y'}{z'}
    \\\\
    &\Rightarrow\ip{x'}{z'} + \ip{y'}{z'} = \ip{x'+y'}{z'}
  \end{align*}
  \\
  \fbox{P4} Given \(x, y, \in E\), define \(S = \{\lambda \in \R
  \enspace | \enspace \lambda\ip{x}{y} = \ip{\lambda x}{y}\}\).  Let
  us proceed to prove that \(S = \R\).
  \begin{description}

  \item[(\(\Z \subset S\))] Clearly \(1\ip{x}{y}=\ip{1x}{y}=\ip{x}{y}\) and thus
    \(1 \in S\).  Suppose that \(\alpha, \beta \in S\), then
    \begin{align*}
      (\alpha\pm\beta)\ip{x}{y} &= \alpha\ip{x}{y}\pm\beta\ip{x}{y}
        \comment{=}{H.} \ip{\alpha x}{y}\pm\ip{\beta x}{y} \\
      & \comment{=}{P3} \ip{\alpha x \pm \beta x}{y}
        = \ip{(\alpha\pm\beta)x}{y}
    \end{align*}
    Which means that \(\alpha\pm\beta\in S\), and therefore \(\Z \subset S\).

  \item[(\(\Q \subset S\))] Suppose that \(\alpha, \beta \in S\) and
    \(\beta \neq 0\). Hence,
    \begin{align*}
      \alpha\ip{x}{y} &\comment{=}{H.} \ip{\alpha x}{y} =
      \ip{\frac{\beta}{\beta}\alpha x}{y}
      \comment{=}{H.} \beta\ip{\frac{\alpha}{\beta}x}{y} \\
      &\Rightarrow \frac{\alpha}{\beta}\ip{x}{y} = \ip{\frac{\alpha}{\beta}x}{y}
    \end{align*}
    Which means that \(\frac{\alpha}{\beta} \in S\), and therefore \(\Q
    \subset S\).

  \item[(\(\R \subset S\))] Given \(x, y \in E\), let \(f\) and \(g\)
    be real functions such that \(f(\lambda)=\lambda\ip{x}{y}\) and
    \(g(\lambda)=\ip{\lambda x}{y}\), for all \(\lambda \in \R\).
    Considering the previous result, it is clear that
    \(f(\lambda)=g(\lambda), \forall \lambda\in\Q\).  Furthermore,
    both functions are continuous because \(f\) is linear and \(g\) is
    composition of continuous functions.  This means that \(f=g\),
    since two real continuous functions whose values coincide for
    every rational number must coincide for each irrational number as
    well.  Therefore \(\R \subset S\).
  \end{description}
\end{proof}

\begin{thebibliography}{9}
\bibitem{JN} P. Jordan and J. v. Neumann; \emph{On inner products in
    linear, metric spaces}; Annals of Mathematics, Vol. 36, No. 3;
  July, 1935.
\end{thebibliography}

\end{document}
